% !TeX program  = lualatex
% !TeX encoding = UTF-8 Unicode
%-----------------------------------------------------------------------------
% 									This is summa-sample-letter-en.tex
% 									A current version can be downloaded from
%									https://github.com/igreil/summa
%
% © Ingmar Greil, 2022. 			The LPPL applies. See summa.sty for details.
%-----------------------------------------------------------------------------
\documentclass[fontsize=11pt,parskip=full]{scrlttr2}
	\pagestyle{empty}				% Suppress page header and footer
	\KOMAoptions{enlargefirstpage,
				parskip=full}		% No paragraph indentation, done correctly
%-----------------------------------------------------------------------------
\setkomavar{fromname}		{Ole does it Åll, Ltd.}
\setkomavar{fromaddress}	{Hersnapvej 41, 1092 København K}
%-----------------------------------------------------------------------------
\setkomavar{date}			[Date]			{\today}
\setkomavar{customer}		[Due]			{\duedate}			% => #86
\setkomavar{yourref}		[Number]		{\invoicenumber{}–\the\year}
\setkomavar{subject}						{Invoice \#\invoicenumber}
%-----------------------------------------------------------------------------
 
\setkomavar{firsthead}{%				% FIRST LINE OF MASTHEAD
	\begin{center}{%
		\fontsize{38}{40}\selectfont	% Switch to font size 38pt for header
		\color{mycolor!50}				% The chosen color, half as colorful
			\textsc{Ole}				% => #48
		\color{mycolor}					% Regular, full hue again
			\textsc{does it åll, ltd.}
	}\\[-.35em]{%
		\color{mycolor!75}				% Let's try ¾
			\rule{\textwidth}{.75pt}
	}
	\end{center}
%-----------------------------------------------------------------------------
	\begin{flushright}					% SECOND LINE OF MASTHEAD
		\color{mycolor}
			\vspace{-3em}\par
			Hersnapvej 41, 1092 København K • www.example.dk\\
			\emph{Så er den ged barberet\rlap{.}}
	
	\end{flushright}
}%----------------------------------------------------------------------------
\usepackage[american]{babel} % Your language of choice, effects a few packages
%-----------------------------------------------------------------------------
\usepackage[forget]{qrcode}					% For QR code (no caching)
\usepackage[absolute]{textpos}				% Absolute positioning of QR code
\usepackage{tcolorbox, xcolor, graphicx}	% Misc. graphics packages
	\definecolor{mycolor}{rgb}{.65,.25,.25}	% See xcolor docs for more
%-----------------------------------------------------------------------------
\usepackage{xstring}						% For string manipulation
\usepackage{lua-ul}							% For underlining (soul replacement)
\usepackage{scrdate,advdate}				% For calculating the due day/date
\usepackage{fmtcount}						% For spelling out the days
%-----------------------------------------------------------------------------
\usepackage{fontspec}						% May I suggest a nice sans font?

% Usual font options for numbers include Monospaced/Proportional 
% and OldStyle/Lining. The latter is a matter of taste and personal 
% preference, but I highly recommend Monospaced for tabular content like 
% invoices. Note that not all fonts support all four combinations.

	\setsansfont[Numbers={Monospaced,		Lining}]	{Noto Sans}
	%\setsansfont[Numbers={Monospaced,		OldStyle}]	{Noto Sans}
	
	\defaultfontfeatures{%
		SmallCapsFeatures = {%
			Letters = SmallCaps,
			Numbers = Lining,
			LetterSpace = -.75	% tighter spacing for small caps
		}, 
		Ligatures = {%
			Common, TeX, Required, %Contextual
		},
	}												% No serifs, no siree, Bob!
	\renewcommand{\familydefault}{\sfdefault}		% Sans all the way
%-----------------------------------------------------------------------------
\usepackage{summa}					% <= This is what we're here for :)
		\TaxRate{25}{0}				% For Denmark. Default is Germany's 19/7

\providecommand*\duedate{%			% Advance the current date by x number
	\AdvanceDate[\payperiod]\today	% of days to arrive at the due date
}%----------------------------------------------------------------------------
\begin{document}	\def\payperiod		{3}		% How many days to pay?
					\def\invoicenumber	{44}	% Reference/invoice number
%-----------------------------------------------------------------------------	
	\begin{letter}{%
%
					Ms. C. Ustomer\\
					Over the Rainbow, Suite 221\kern 1pt B\\
					1234 Vienna, Austria\\				
					\vspace{.25em}\par		% Skip a bit, brother, and
											% flush mail address to the right
					\underLine{sent by E-Mail} \hfill\emph{customer@example.org}
	}
%-----------------------------------------------------------------------------	
		\opening{Dear Ms. Ustomer,}

		\textsc{Ole does it Åll, ltd.} appreciates your business.

		\begin{invoice}[N]				% N = Netto			(Net)
										% B = Brutto		(Gross)
			\NumbersOff					% X = Ohne Steuer	(No Taxes)
			\SeparatorOff


% This needs to be more userfriendly:
\makeatletter
	\renewcommand*\@amount		{~}					% Item amount
	\renewcommand*\@description	{Description}		% Item description
	\renewcommand*\@unitprice	{Unit price}		% Item price
	\renewcommand*\@totalprice	{Total price}		% Item price × amount
%	\renewcommand*\@decimalsep	{.}					% decimal separator

	\renewcommand*\@taxabrv		{Combined Sales Tax}
	\renewcommand*\@sumnet		{Subtotal}	% Net Total, before taxes
	\renewcommand*\@sumtot		{Total}	% Gross Total, with taxes
	\renewcommand*\@excl		{}	% excl.

\makeatother
										% The actual invoice part:
%=============================================================================
	%\InvoiceLine{How many}	{Description}	{Unit price} 
	% [e] is for r*E*duced tax rate (but really should be \InvoiceLine*)
\InvoiceLine[e]{1}	{All The News That's Fit To Print\,™}		{2,99}
\InvoiceLine{2}		{Urinacanis Pale Ale		\tab 500 ml}	{22,90}
\InvoiceLine{1}		{Pizza XXL					\tab 45 cm}		{19,95}
\InvoiceLine{1}		{Pizza S					\tab 12 cm}		{9,95}
%=============================================================================

		\end{invoice}

	% If you need an article number, use the starred environment:
	%	\begin{invoice*}
	%		\InvoiceLine{How many}{Article number}{Description}{Unit price} 
	%	\end{invoice*}

	\marginpar{\textbf{FIXME} •\;German}	%FIXME

We are looking forward to receiving the full amount of
 \Total\ (including \TaxAmnt\ in taxes) within \numberstringnum{\payperiod}
 days{}, i.e. on or before \DayName{\year}{\month}{\day+\payperiod},
 \AdvanceDate[\payperiod]\today{}. 

		\AdvanceDate[-\payperiod]				% Restore current date

% This method of payment is only available for SEPA payments, i.e. within the 
% "Single Euro Payment Area". No other standards are supported. 

% Switzerland has it own, incompatible standard. 
%	=> https://www.ctan.org/pkg/qrbill

% There might still be a use for an English language template, but if it 
% involes transactions outside of SEPA – even in Euro – the QR code below 
% will be of little to no use to you.

\begin{textblock*}{10cm}(13.5cm,4.5cm)			% Absolutely positioned QR code
	\begin{minipage}[c]{6,75cm}
%		Draw a border around the QR code, with rounded corners:
		\begin{tcolorbox}[colback=white,grow to right by=-17mm]

%	A few definitions for the payment QR code:
	\def\creditor	{Pizzeria Copy Shop}
	\def\reference	{Invoice\ \#\ \invoicenumber/\the\year}
	\def\bic		{PBNKDEFF}					% => Postbank
	\def\iban		{DE02100100100006820101}	% valid test IBAN
	
% FIXME Need to strip \Total of the € sign and feed the raw value.
% For now we have to hardcode the amount :-(

	% Both . and , are accepted decimal separators
	% Full amounts without any are fine, too.
	
	\def\amount		{2}							% Final amount in EUR

% No further comments, linebreaks, spaces, special characters etc. below. 
% Very finicky. Do not touch!
%=============================================================================
\qrcode[height=4cm]{BCD\?001\?1\?SCT\?\bic\?\creditor\?\iban\?EUR\amount\?\?\reference\?\?}
%=============================================================================
		\end{tcolorbox}
	\end{minipage}
\end{textblock*}

		% Compliance with Austrian standards, safe to comment out or modify:
		\begin{textblock*}{2cm}(18.15cm,5.75cm)%
%			\rotatebox{90}{\colorbox{white}{Bezahlen mit Code}}
			\rotatebox{90}{\colorbox{white}{Scan to Pay}}
		\end{textblock*}

\setkomavar{signature}{}	% No signature
\closing{}					% No closing

	\end{letter}
\end{document}
% EOF