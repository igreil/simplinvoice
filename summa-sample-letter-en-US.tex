% !TeX program  = lualatex
% !TeX encoding = UTF-8 Unicode
%-----------------------------------------------------------------------------
% 									This is summa-sample-letter-en.tex
% 									A current version can be downloaded from
%									https://github.com/igreil/summa
%
% © Ingmar Greil, 2022. 			The LPPL applies. See summa.sty for details.
%-----------------------------------------------------------------------------
\documentclass[	backaddress=off,
				paper=letter,
				fontsize=11pt,
				parskip=full]{scrlttr2}
	\pagestyle{empty}				% Suppress page header and footer
	\KOMAoptions{enlargefirstpage,
				parskip=full}		% No paragraph indentation, done correctly
%-----------------------------------------------------------------------------
\setkomavar{fromname}		{Leo's Pizza \& Stuff}
\setkomavar{fromaddress}	{3316 Scheuvront Drive, Denver, CO 80231, USA}
%-----------------------------------------------------------------------------
\setkomavar{date}			[Date]			{\today}
\setkomavar{customer}		[Due]			{\duedate}			% => #86
\setkomavar{yourref}		[Number]		{\invoicenumber{}–\the\year}
\setkomavar{subject}						{Invoice \#\invoicenumber}
%-----------------------------------------------------------------------------
\setkomavar{firsthead}{%				% FIRST LINE OF MASTHEAD
	\begin{center}{%
		\fontsize{38}{40}\selectfont	% Switch to font size 38pt for header
		\color{mycolor!50}				% The chosen color, half as colorful
			\textsc{Leo's}				% => #48
		\color{mycolor}					% Regular, full hue again
			\textsc{Pizzas \& Stuff}
	}\\[-.35em]{%
		\color{mycolor!75}				% Let's try ¾
			\rule{\textwidth}{.75pt}
	}
	\end{center}
%-----------------------------------------------------------------------------
	\begin{flushright}					% SECOND LINE OF MASTHEAD
		\color{mycolor}
			\vspace{-3em}\par
			3316 Scheuvront Drive, Denver, CO 80231 • www.example.com\\
			Proudly serving your late night cravings since 1991
	\end{flushright}
}%----------------------------------------------------------------------------
\usepackage[american]{babel} % Your language of choice, effects a few packages
%-----------------------------------------------------------------------------
\usepackage[forget]{qrcode}					% For QR-code (no caching)
\usepackage[absolute]{textpos}				% Absolute positioning of QR-code
\usepackage{tcolorbox, xcolor, graphicx}	% Misc. graphics packages
	\definecolor{mycolor}{rgb}{.25,.65,.25}	% See xcolor docs for more
%-----------------------------------------------------------------------------
\usepackage{xstring}						% For string manipulation
\usepackage{lua-ul}							% For underlining (soul replacement)
\usepackage{scrdate,advdate}				% For calculating the due day/date
\usepackage{fmtcount}						% For spelling out the days
%-----------------------------------------------------------------------------
\usepackage{fontspec}						% May I suggest a nice sans font?

% Usual font options for numbers include Monospaced/Proportional 
% and OldStyle/Lining. The latter is a matter of taste and personal 
% preference, but I highly recommend Monospaced for tabular content like 
% invoices. Note that not all fonts support all four combinations.

	\setsansfont[Numbers={Monospaced,		Lining}]	{Noto Sans}
	%\setsansfont[Numbers={Monospaced,		OldStyle}]	{Noto Sans}
	
	\defaultfontfeatures{%
		SmallCapsFeatures = {%
			Letters = SmallCaps,
			Numbers = Lining,
			LetterSpace = -.75	% tighter spacing for small caps
		}, 
		Ligatures = {%
			Common, TeX, Required, %Contextual
		},
	}												% No serifs, no siree, Bob!
	\renewcommand{\familydefault}{\sfdefault}		% Sans all the way
%-----------------------------------------------------------------------------
\usepackage{summa}					% <= This is what we're here for :)
		\TaxRate{9}{0}			% For Denver, CO. Default is Germany's 19/7
									% It's actual 8.81%, but we can't do 
									% anything but natural numbers for now.

\providecommand*\duedate{%			% Advance the current date by x number
	\AdvanceDate[\payperiod]\today	% of days to arrive at the due date
}%----------------------------------------------------------------------------
\begin{document}	\def\payperiod		{14}	% How many days to pay?
					\def\invoicenumber	{42}	% Reference/invoice number
%-----------------------------------------------------------------------------	
	\begin{letter}{%
%
					Ms. C. Ustomer\\
					Over the Rainbow, Suite 221\kern 1pt B\\
					1234 Vienna, Austria\\				
					\vspace{.25em}\par		% Skip a bit, brother, and
											% flush mail address to the right
					\underLine{Sent by E-Mail} \hfill\emph{customer@example.org}
	}
%-----------------------------------------------------------------------------	
		\opening{Dear Ms. Ustomer,}

		\textsc{Leo's Pizzas \& Stuff} appreciates your business.%
		\marginpar{\textbf{FIXME} •\;Currency •\;German •\;Decimals}	%FIXME

		\begin{invoice}[N]				% N = Netto			(Net)
										% B = Brutto		(Gross)
			\NumbersOff					% X = Ohne Steuer	(No Taxes)
			\SeparatorOff
			\UseDollar

% This needs to be more userfriendly:
\makeatletter
	\renewcommand*\@amount		{~}					% Item amount
	\renewcommand*\@description	{Description}		% Item description
	\renewcommand*\@unitprice	{Unit price}		% Item price
	\renewcommand*\@totalprice	{Total price}		% Item price × amount
	\renewcommand*\@decimalsep	{.}					% decimal separator

	\renewcommand*\@taxabrv		{Combined Sales Tax}
	\renewcommand*\@sumnet		{Subtotal}	% Net Total, before taxes
	\renewcommand*\@sumtot		{Total}	% Gross Total, with taxes
	\renewcommand*\@excl		{}	% excl.

\makeatother
										% The actual invoice part:
%=============================================================================
	%\InvoiceLine{How many}	{Description}	{Unit price} 
	% [e] is for r*E*duced tax rate (but really should be \InvoiceLine*)
\InvoiceLine{48}	{Urinacanis Pale Ale		\tab 16 oz.}	{2.79}
\InvoiceLine{21}	{Koala-Kola, sweet-sour 	\tab 12 oz.}	{1.59}
\InvoiceLine{8}		{Large Pizza}								{12.25}
\InvoiceLine{250}	{Misc. Printing}							{1.99}
%=============================================================================

		\end{invoice}

	% If you need an article number, use the starred environment:
	%	\begin{invoice*}
		%		\InvoiceLine{How many}{Article number}{Description}{Unit price} 
	%	\end{invoice*}

We are looking forward to receiving the full amount of
 \Total\ (including \TaxAmnt\ in taxes) within \numberstringnum{\payperiod}
 days{}, i.e. on or before \DayName{\year}{\month}{\day+\payperiod},
 \AdvanceDate[\payperiod]\today{}. 

		\AdvanceDate[-\payperiod]				% Restore current date

\setkomavar{signature}{}	% No signature
\closing{}					% No closing

	\end{letter}
\end{document}
% EOF