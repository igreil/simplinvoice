% !TeX program  = lualatex
% !TeX encoding = UTF-8 Unicode
%-----------------------------------------------------------------------------
% 									This is summa-sample-letter-de.tex
% 									A current version can be downloaded from
%									https://github.com/igreil/summa
%
% © Ingmar Greil, 2022. 			The LPPL applies. See summa.sty for details.
%-----------------------------------------------------------------------------
\documentclass[fontsize=11pt,parskip=full]{scrlttr2}
	\LoadLetterOption{DIN}			% Some default values for German letters
	\pagestyle{empty}				% Suppress page header and footer
	\KOMAoptions{enlargefirstpage,
				parskip=full}		% No paragraph indentation, done correctly
%-----------------------------------------------------------------------------
\setkomavar{fromname}		{Peter Rokurist}
\setkomavar{fromaddress}	{Some Street 22/7, 1000 Wien}
%-----------------------------------------------------------------------------
\setkomavar{date}			[Datum]			{\today}
\setkomavar{customer}		[Fälligkeit]	{\duedate}			% => #86
\setkomavar{yourref}		[Nummer]		{\invoicenumber}	% => #91
\setkomavar{subject}						{Rechnung Nr.~\invoicenumber}
%-----------------------------------------------------------------------------
\setkomavar{firsthead}{%				% FIRST LINE OF MASTHEAD
	\begin{center}{%
		\fontsize{38}{40}\selectfont	% Switch to font size 38pt for header
		\color{mycolor!50}				% The chosen color, half as colorful
			\textsc{Pizzeria}
		\color{mycolor}					% Regular, full hue again
			\textsc{\& Copy Shop}
	}\\[-.35em]{%
		\color{mycolor!75}				% Let's try ¾
			\rule{\textwidth}{.75pt}
	}
	\end{center}
%-----------------------------------------------------------------------------
	\begin{center}						% SECOND LINE OF MASTHEAD
		\color{mycolor}					% Separate environment, so you 
			\vspace{-3.5em}\par			% could also flushright/-left easily
			Some Street 22/7, 1000 Wien • www.example.com
			 • DE02\;1001\;0010\;0006\;8201\;01		% \; = .2777em space
	\end{center}						% Separate environment, so you 
										% could also flushright/-left easily
}%----------------------------------------------------------------------------
\usepackage[ngerman]{babel}	% Your language of choice, effects a few packages
%-----------------------------------------------------------------------------
\usepackage[forget]{qrcode}					% QR-codes, no caching
\usepackage[absolute]{textpos}				% Absolute positioning of QR-codes
\usepackage{tcolorbox, xcolor, graphicx}	% Misc. graphics packages
	\definecolor{mycolor}{rgb}{.25,.25,.9}	% See xcolor docs for more
%-----------------------------------------------------------------------------
\usepackage{xstring}						% For string manipulation
\usepackage{lua-ul}							% For underlining (soul replacement)
\usepackage{scrdate,advdate}				% For calculating the due day/date
\usepackage{fmtcount}						% For spelling out the days
%-----------------------------------------------------------------------------
\usepackage{fontspec}						% I suggest a nice Sans font

% Usual font options for numbers include Monospaced/Proportional 
% and OldStyle/Lining. The latter is a matter of taste and personal 
% preference, but I highly recommend Monospaced for tabular content like 
% invoices. Note that not all fonts support all four combinations.

	\setsansfont[Numbers={Monospaced,		Lining}]	{Noto Sans}
	%\setsansfont[Numbers={Monospaced,		OldStyle}]	{Noto Sans}
	
	%\setsansfont[Numbers={Proportional,	Lining}]	{Noto Sans}
	%\setsansfont[Numbers={Proportional,	OldStyle}]	{Noto Sans}

	\defaultfontfeatures{%
		SmallCapsFeatures = {%
			Letters = SmallCaps,
			Numbers = Lining,
			LetterSpace = -.75	% tighter spacing for small caps
		}, 
		Ligatures = {%
			Common, TeX, Required, %Contextual
		},
	}												% No serifs, no siree, Bob!
	\renewcommand{\familydefault}{\sfdefault}		% Sans all the way
%-----------------------------------------------------------------------------
\usepackage{summa}					% <= This is what we're here for :)
		\TaxRate{20}{10}			% For Austria. Default is Germany's 19/7
									% Only natural numbers for now, please.
\providecommand*\duedate{%			% Advance the current date by x number
	\AdvanceDate[\payperiod]\today	% of days to arrive at the due date
}%----------------------------------------------------------------------------
\begin{document}	\def\payperiod		{14}	% How many days to pay?
					\def\invoicenumber	{42}	% Reference/invoice number
%-----------------------------------------------------------------------------	
	\begin{letter}{%
%
					Ms. C. Ustomer\\
					Over the Rainbow, Suite 200c\\
					1234 Wien\\				
					\vspace{.25em}\par		% Skip a bit, brother, and
											% flush mail address to the right
					\underLine{per E-Mail} \hfill\emph{customer@example.org}
	}
%-----------------------------------------------------------------------------	
		\opening{Sehr geehrte Damen und Herren\kern 1pt !}

		Die \textsc{Pizzeria \& Copy Shop GmbH} dankt sehr herzlich 
		für Ihren Auftrag.

		\begin{invoice*}[B]				% N = Netto			(Net)
										% B = Brutto		(Gross)
			\NumbersOff					% X = Ohne Steuer	(No Taxes)
			\SeparatorOff
										% The actual invoice part:
%=============================================================================
	%\InvoiceLine{How many}{Article number}{Description}{Unit price} 
	% [e] is for r*E*duced tax rate (but really should be \InvoiceLine*)
\InvoiceLine[e]{48}	{47-11}		{Urinacanis Pale Ale\tab 0,50\,ℓ}	{3,60}
\InvoiceLine[e]{21}	{47-12}		{Koala-Kola, süß-sauer \tab 0,33\,ℓ}{2,20}
\InvoiceLine[e]{8}	{47-99}		{Große Pizza}						{15,50}
\InvoiceLine{250}	{08-15}		{Druck Plakate \tab A3}				{1,75}
\InvoiceLine{2,5}	{---~~~}	{Arbeitsstunden Praktikant}			{25,50}
%=============================================================================

		\end{invoice*}

	% Replace the ℓ sign (for liter) in lines 114f with "L" or "l" if you 
	% are not a fan, or your font simply doesn't have the glyph.

	% If you don't need an article number, use the un-starred environment 
	% 	\begin{invoice} … \end{invoice}

		Bitte überweisen Sie den offenen Betrag von 
		\Total\ (darin enthalten \TaxAmnt ~\TaxAbrv) 
		binnen \numberstringnum{\payperiod}~Tagen – bis \DayName{\year}{\month}%
		{\day+\payperiod}, den \AdvanceDate[\payperiod]\today\ – 
		auf unser Konto bei der X-Bank.

		\AdvanceDate[-\payperiod]				% Restore current date

\begin{textblock*}{10cm}(13.5cm,4.5cm)			% Absolutely positioned QR code
	\begin{minipage}[c]{6,75cm}
%		Draw a border around the QR Code, with rounded corners:
		\begin{tcolorbox}[colback=white,grow to right by=-17mm]

%	A few definitions for the payment QR-code:
	\def\creditor	{Pizzeria Copy Shop}
	\def\reference	{Rechnungsnr.\ \invoicenumber/\the\year}
	\def\bic		{PBNKDEFF}				% => Postbank
	\def\iban		{DE02100100100006820101}% valid test IBAN
	\def\currcode	{EUR}					% probably the only sensible choice

% FIXME Need to strip \Total of the € sign and feed the raw value.
% For now we have to hardcode the amount :-(

	% Both . and , are accepted decimal separators
	% Full amounts without them are fine, too.
	\def\amount		{1.20}							% amount in \currcode

% No further comments, linebreaks, spaces, special characters etc. below. 
% Very finicky. Do not touch!
%=============================================================================
\qrcode[height=4cm]{BCD\?001\?1\?SCT\?\bic\?\creditor\?\iban\?\currcode\amount\?\?\reference\?\?}
%=============================================================================
		\end{tcolorbox}
	\end{minipage}
\end{textblock*}

		% Conforming to Austrian standards, safe to comment out or modify:
		\begin{textblock*}{2cm}(18.15cm,5.75cm)%
			\rotatebox{90}{\colorbox{white}{Zahlen mit Code}}
		\end{textblock*}

		\setkomavar{signature}{%
					\hspace{75mm}P. Rokurist{}~p.\kern 1pt p.\kern 1pt a
		}
		\closing{Mit freundlichen Grüßen}

	\end{letter}
\end{document}
% EOF