% !TeX program  = lualatex
% !TeX encoding = UTF-8 Unicode
%-----------------------------------------------------------------------------
%
% 				This is summa-sample-invoices.tex
% 				A current version can be downloaded from
%				https://github.com/igreil/summa
%
%-----------------------------------------------------------------------------
% This document implements various invoices … FIXME
%-----------------------------------------------------------------------------
\documentclass[12pt]{scrartcl}		% "article" could probably be made to work
	\pagestyle{empty}				% Suppress page header and footer
	\setkomafont{section}{\LARGE}	% Font size of title
	\KOMAoptions{parskip=full}		% No paragraph indentation
%-----------------------------------------------------------------------------
\usepackage[ngerman]{babel}			% Your language of choice
%-----------------------------------------------------------------------------
\usepackage{advdate,scrdate}		% For calculating the due date
\usepackage{fontspec}				% Select font with tabular/lining figures:
	\setsansfont[Numbers={Monospaced,Lining}]{Source Sans Pro}
	\renewcommand{\familydefault}{\sfdefault}		% Sans all the way
%-----------------------------------------------------------------------------
\usepackage{../summa}					% <= This is what we're here for :)
		\TaxRate{20}{10}			% For Austria. Default is Germany's 19/7
									% Only natural numbers for now, please.
%-----------------------------------------------------------------------------
\begin{document}
\section*{Rechnung Nr.\,42/2022\hfill\today}	% Starred => unnumbered section

Die \textsc{Pizzeria \& Copy Shop GmbH} dankt sehr herzlich für Ihren Auftrag:

%-----------------------------------------------------------------------------
	\begin{invoice*}[B]				% N = Netto			(Net)
									% B = Brutto		(Gross)
									% X = Ohne Steuer	(No Taxes)
%-----------------------------------------------------------------------------
%		\UseEuro					% The Euro (€) is the default currency.
		\NumbersOff					% Running item numbers? Your choice.
		\SeparatorOff				% Horizontal lines? Heck, no.
%-----------------------------------------------------------------------------
		% All prices can have up to (!) two decimal places.
		% Both commas and periods are fine. Leading zeros may 
		% be omitted, even though it looks decidedly ugly.
%-----------------------------------------------------------------------------
		\InvoiceLine[e]{2}{4711}{Urinacanis Pale Ale\tab 0,50\,ℓ}		{3,60}
		\InvoiceLine[e]{12}{4712}{Koala-Kola, süß-sauer \tab 0,33\,ℓ}	{2,20}
		\InvoiceLine[e]{8}{4799}{Große Pizza}							{15,50}
		\InvoiceLine{20}{0815}{Druck Plakate \tab A3}					{2,50}
			% Decimal values are allowed, in principle:
		\InvoiceLine{2,25}{---~~~}{Arbeitsstunden Praktikant}			{25,50}
		\Shipping{3,90}
		\Credit{Anzahlung}{120,00}
	\end{invoice*}
%-----------------------------------------------------------------------------
	\def\payperiod{8}			% How many days to pay?

Bitte überweisen Sie den offenen Betrag von 
\Total\ (darin enthalten \TaxAmnt ~\TaxAbrv) 
binnen \payperiod~Tagen – bis \DayName{\year}{\month}{\day+\payperiod}, 
den \AdvanceDate[\payperiod]\today\ – 
auf unser Konto DE75\;5121\;0800\;1245\;1261\;99.

\AdvanceDate[-\payperiod]		% Restore current date

\section*{Rechnung Nr.\,43/2022\hfill\today}
	\begin{invoice*}[N]				% N = Netto			(Net)
									% B = Brutto		(Gross)
									% X = Ohne Steuer	(No Taxes)
%-----------------------------------------------------------------------------
		%\SetCurrency{₹} 			% Indian Rupees? Sure.
		\NumbersOff					% Running item numbers? Your choice.
		\SeparatorOff				% Horizontal lines? Heck, no.
%-----------------------------------------------------------------------------
		\InvoiceLine{1}{~}{Leisure Suit, Mod. “Matris Fututor”}		{2249,90}
		\InvoiceLine{3,5}{hrs}{Misc. Sewing Work}					{68}
	\end{invoice*}

\def\payperiod{21}					% How many days to pay?

Bitte überweisen Sie den offenen Betrag von 
\Total\ (darin enthalten \TaxAmnt ~\TaxAbrv) 
binnen \payperiod~Tagen – bis \DayName{\year}{\month}{\day+\payperiod}, 
den \AdvanceDate[\payperiod]\today\ – 
auf unser Konto DE75\;5121\;0800\;1245\;1261\;99.

\AdvanceDate[-\payperiod]		% Restore current date

%-----------------------------------------------------------------------------
\AdvanceDate[+1]				% Invoice with tomorrow's date

\clearpage
\section*{Rechnung Nr.\,44/2022\hfill\today}

	\begin{invoice}[N]				% N = Netto			(Net)
									% B = Brutto		(Gross)
									% X = Ohne Steuer	(No Taxes)
%-----------------------------------------------------------------------------
		\NumbersOff					% Running item numbers? Your choice.
		\SeparatorOff				% Horizontal lines? Heck, no.
%-----------------------------------------------------------------------------
		\InvoiceLine{1000}{Kopien \tab A4}						{0,18}
	\end{invoice}

\def\payperiod{3}					% How many days to pay?

Bitte überweisen Sie den offenen Betrag von 
\Total\ (darin enthalten \TaxAmnt ~\TaxAbrv) 
binnen \payperiod~Tagen – bis \DayName{\year}{\month}{\day+\payperiod}, 
den \AdvanceDate[\payperiod]\today\ – 
auf unser Konto DE75\;5121\;0800\;1245\;1261\;99.

\AdvanceDate[-\payperiod]		% Restore current date
%-----------------------------------------------------------------------------
\AdvanceDate[+7]				% Invoice with tomorrow's date

\section*{Rechnung Nr.\,45/2022\hfill\today}

	\begin{invoice}[X]				% N = Netto			(Net)
									% B = Brutto		(Gross)
									% X = Ohne Steuer	(No Taxes)
%-----------------------------------------------------------------------------
		\NumbersOff					% Running item numbers? Your choice.
		\SeparatorOff				% Horizontal lines? Heck, no.
%-----------------------------------------------------------------------------
		\InvoiceLine{1}{Barauslage}						{532,00}
	\end{invoice}

\def\payperiod{7}					% How many days to pay?

Bitte überweisen Sie den offenen Betrag von 
\Total\ (darin enthalten \TaxAmnt ~\TaxAbrv) 
binnen \payperiod~Tagen – bis \DayName{\year}{\month}{\day+\payperiod}, 
den \AdvanceDate[\payperiod]\today\ – 
auf unser Konto DE75\;5121\;0800\;1245\;1261\;99.

\AdvanceDate[-\payperiod]		% Restore current date

%-----------------------------------------------------------------------------

\section*{Invoice Nr.\,46/2022\hfill\today}
\TabPositions{.62\linewidth}	% Sets a tabstop

\makeatletter
	\renewcommand*\@running		{Nr.}				% Number
	\renewcommand*\@amount		{Amount}			% Item amount
	\renewcommand*\@itemnr		{~}					% Item number
	\renewcommand*\@description	{Description}		% Item description
	\renewcommand*\@unitprice	{Base Fee}			% Item price
	\renewcommand*\@totalprice	{Total}				% Item price × amount
	\renewcommand*\@decimalsep	{.}					% decimal separator
\makeatother

\UseDollar
		\TaxRate{20}{0}			
	\begin{invoice*}[B]				% N = Netto			(Net)
									% B = Brutto		(Gross)
									% X = Ohne Steuer	(No Taxes)
%-----------------------------------------------------------------------------
		\NumbersOff					% Running item numbers? Your choice.
		\SeparatorOff				% Horizontal lines? Heck, no.
%-----------------------------------------------------------------------------
		\InvoiceLine[e]{1}{~}{Administrative Fee \tab(flat)}			{50.00}
		\InvoiceLine{8.3}{hrs}{Legal Research \tab(Associate)}		{250.00}
		\InvoiceLine{.5}{hrs}{Motion drafted \tab(Junior Partner)}	{400.00}
		\InvoiceLine{.4}{hrs}{Motion reviewed and filed \tab(Senior Partner)}	{655.00}
	\end{invoice*}

\def\payperiod{14}					% How many days to pay?
\enlargethispage{2\baselineskip}

Thank you for patronizing the Law Offices of \textsc{Dewe, Cheatem \& Howe}. Now, please pay up!
\end{document}
% EOF