%-----------------------------------------------------------------------------
%\pagestyle{empty}					% Suppress page header and footer
	\KOMAoptions{	enlargefirstpage,
					parskip=full}	% No paragraph indentation, done correctly

\usepackage[ngerman]{babel}	% Your language of choice, effects a few packages
%-----------------------------------------------------------------------------
\usepackage[forget]{qrcode}					% QR codes, no caching
\usepackage[absolute]{textpos}				% Absolute positioning of QR codes
\usepackage{tcolorbox, xcolor, graphicx}	% Misc. graphics packages
	\definecolor{mycolor}{rgb}{.25,.25,.9}	% See xcolor docs for more
%-----------------------------------------------------------------------------
\usepackage{xstring}						% For string manipulation
\usepackage{lua-ul}							% For underlining (soul replacement)
\usepackage{scrdate,advdate}				% For calculating the due day/date
\usepackage{fmtcount}						% For spelling out the days
%-----------------------------------------------------------------------------
\usepackage{fontspec}						% I suggest a nice Sans font

% Usual font options for numbers include Monospaced/Proportional 
% and OldStyle/Lining. The latter is a matter of taste and personal 
% preference, but I highly recommend Monospaced for tabular content like 
% invoices. Note that not all fonts support all four combinations.

	\setsansfont[Numbers={Monospaced,		Lining}]	{Noto Sans}
	%\setsansfont[Numbers={Monospaced,		OldStyle}]	{Noto Sans}
	
	\defaultfontfeatures{%
		SmallCapsFeatures = {%
			Letters = SmallCaps,
			Numbers = Lining,
			LetterSpace = -.75	% Tighter spacing for small caps
		}, 
		Ligatures = {%
			Common, TeX, Required, Contextual
		},
	}												% No serifs, no siree, Bob!
	\renewcommand{\familydefault}{\sfdefault}		% Sans all the way
%-----------------------------------------------------------------------------
\usepackage{summa}					% <= This is what we're here for :)
		\TaxRate{20}{10}			% For Austria. Default is Germany's 19/7
									% Only natural numbers for now, please.
\providecommand*\duedate{%			% Advance the current date by x number
	\AdvanceDate[\payperiod]\today	% of days to arrive at the due date
}%----------------------------------------------------------------------------
\providecommand*\printservicedate{%
	\ifx\servicedate\empty\AdvanceDate[-\daysago]\today
	\else\servicedate\fi
}%----------------------------------------------------------------------------